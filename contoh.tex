\documentclass{ta-its}

% \title{Judul Bahasa Indonesia}{Judul Bahasa Inggris}
\title{Templat \LaTeX{} untuk Kebutuhan Penulisan Buku Tugas Akhir di ITS Surabaya}{A \LaTeX{} Template for Setting Up a Final Project Booklet for ITS Surabaya}{Kxxxxx} 

% \author{Nama Lengkap}{NRP}
\author{Putu Wiramaswara Widya}{5111100012}

% \supervisorOne{Nama Pembimbing Satu}{NIP}
% \supervisorTwo{Nama Pembimbing Dua}{NIP}
\supervisorOne{Royyana Muslim Ijtihadie, S.Kom, M.Kom, PhD}{197708242006041001}

% \degree{Nama Gelar}{Bidang Studi}{Program Studi}{Jurusan}{Jurusan (English)}{Fakultas}{Fakultas Singkatan}{Fakultas (English)}
\degree{Sarjana Komputer}{Arsitektur dan Jaringan Komputer}{S1}{Teknik Informatika}{Informatics}{Teknologi Informasi}{FTIf}{Information Technology}

% \time{bulan}{tahun}
\time{Januari}{2015}

\begin{document}
    \frontmatter % Halaman dengan penomoran romawi kecil
    \maketitle
    %\legalityPaper % Lembar Pengesahan
    \chapter{Kata Pengantar}
        \textbf{Om Swastyastu}

        Puji syukur penulis haturkan kepada Ida Sang Hyang Widhi Wasa, Tuhan Yang Maha Esa karena atas \emph{asungkertha wara nugraha} beliau, penulis dapat menyelesaikan sebuah dokumentasi cara pembuatan Buku Tugas Akhir Sarjana menggunakan \LaTeX{} untuk Institut Teknologi Sepuluh Nopember, Surabaya. Dokumentasi ini diharapkan dapat membantu rekan-rekan mahasiswa S1 yang menempuh semester terakhir dengan membuat buku Tugas Akhir menggunaan sistem \emph{typesetting} \LaTeX{} yang terbukti handal dan lumrah digunakan di bidang penelitian sains dan teknik. Dokumentasi ini dibuat menggunakan templat yang penulis buat sendiri (pada berkas \texttt{ta-its.cls}) sehingga nantinya bisa digunakan kembali sehingga pembuatan buku bisa lebih dipermudah.

        Penulis menerima kritik dan saran mengenai pengembangan templat ini agar bisa menjadi lebih baik dan bisa menjadi standar \emph{de-facto} dan \emph{de-jure} dalam penulisan buku TA di seluruh civitas akademika ITS. Penulis dapat dihubungi melalui surel: \texttt{initrunlevel0@gmail.com}.

        Sekian dan Terima Kasih.
        \noindent \textbf{Om Santhi Santhi Santhi Om}

        \cleardoublepage % Mengisi penanda halaman genap yang kosong

    \tableofcontents % Daftar isi
    \listoftables % Daftar tabel
    \listoffigures % Daftar figur/gambar

\mainmatter % Halaman utama, dengan judul BAB X, nomor halaman penomoran arab
    \chapter{PENDAHULUAN}
        Bab ini membahas mengenai pengenalan \LaTeX{} dalam penulisan ilmiah.

        \section{Latar Belakang}
            Tidak dipungkuri lagi, dunia komputer sangat berkembang pesat semenjak ditemukannya kakas perhitungan dan kalkulasi canggih ini di abad 21. Komputer bukan hanya sekedar kakas bantu kalkulasi matematika semata, namun juga merupakan alat yang membantu manusia untuk segala hal dalam kehidupannya. Salah satu kegunaan komputer di masa modern saat ini adalah membantu pengguna dalam menyiapkan dokumen teks dari proses penulisan draf, penyuntingan dan penyetakan hasil final menggunakan teknologi yang disebut dengan \emph{word processing}.
            
            Teknologi \emph{word-processing} berkembang semenjak 
        \section{Rumusan Masalah}
        \section{Tujuan}
        \section{Manfaat}
    
    \chapter{PEMBAHASAN}
        \section{Instalasi \LaTeX{}}
        \section{\texttt{Hello World} menggunakan \LaTeX{}}
        \section{Filosofi \LaTeX{}}
        \section{Cara Menggunakan Templat \texttt{ta-its}}
        \section{Struktur Dokumen \LaTeX{}}
        \section{Paragraph dan Teks}
        \section{Daftar}
        \section{Gambar}
        \section{Tabel}
        \section{Rumus Matematika}
        \section{Algoritma}
        \section{Kode Sumber}
        

\appendix % Halaman lampiran, dengan judul LAMPIRAN X

\backmatter % Lampiran tanpa judul LAMPIRAN X, biasanya untuk BIODATA PENULIS
\end{document}
